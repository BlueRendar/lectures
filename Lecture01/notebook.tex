
% Default to the notebook output style

    


% Inherit from the specified cell style.




    
\documentclass[11pt]{article}

    
    
    \usepackage[T1]{fontenc}
    % Nicer default font (+ math font) than Computer Modern for most use cases
    \usepackage{mathpazo}

    % Basic figure setup, for now with no caption control since it's done
    % automatically by Pandoc (which extracts ![](path) syntax from Markdown).
    \usepackage{graphicx}
    % We will generate all images so they have a width \maxwidth. This means
    % that they will get their normal width if they fit onto the page, but
    % are scaled down if they would overflow the margins.
    \makeatletter
    \def\maxwidth{\ifdim\Gin@nat@width>\linewidth\linewidth
    \else\Gin@nat@width\fi}
    \makeatother
    \let\Oldincludegraphics\includegraphics
    % Set max figure width to be 80% of text width, for now hardcoded.
    \renewcommand{\includegraphics}[1]{\Oldincludegraphics[width=.8\maxwidth]{#1}}
    % Ensure that by default, figures have no caption (until we provide a
    % proper Figure object with a Caption API and a way to capture that
    % in the conversion process - todo).
    \usepackage{caption}
    \DeclareCaptionLabelFormat{nolabel}{}
    \captionsetup{labelformat=nolabel}

    \usepackage{adjustbox} % Used to constrain images to a maximum size 
    \usepackage{xcolor} % Allow colors to be defined
    \usepackage{enumerate} % Needed for markdown enumerations to work
    \usepackage{geometry} % Used to adjust the document margins
    \usepackage{amsmath} % Equations
    \usepackage{amssymb} % Equations
    \usepackage{textcomp} % defines textquotesingle
    % Hack from http://tex.stackexchange.com/a/47451/13684:
    \AtBeginDocument{%
        \def\PYZsq{\textquotesingle}% Upright quotes in Pygmentized code
    }
    \usepackage{upquote} % Upright quotes for verbatim code
    \usepackage{eurosym} % defines \euro
    \usepackage[mathletters]{ucs} % Extended unicode (utf-8) support
    \usepackage[utf8x]{inputenc} % Allow utf-8 characters in the tex document
    \usepackage{fancyvrb} % verbatim replacement that allows latex
    \usepackage{grffile} % extends the file name processing of package graphics 
                         % to support a larger range 
    % The hyperref package gives us a pdf with properly built
    % internal navigation ('pdf bookmarks' for the table of contents,
    % internal cross-reference links, web links for URLs, etc.)
    \usepackage{hyperref}
    \usepackage{longtable} % longtable support required by pandoc >1.10
    \usepackage{booktabs}  % table support for pandoc > 1.12.2
    \usepackage[inline]{enumitem} % IRkernel/repr support (it uses the enumerate* environment)
    \usepackage[normalem]{ulem} % ulem is needed to support strikethroughs (\sout)
                                % normalem makes italics be italics, not underlines
    

    
    
    % Colors for the hyperref package
    \definecolor{urlcolor}{rgb}{0,.145,.698}
    \definecolor{linkcolor}{rgb}{.71,0.21,0.01}
    \definecolor{citecolor}{rgb}{.12,.54,.11}

    % ANSI colors
    \definecolor{ansi-black}{HTML}{3E424D}
    \definecolor{ansi-black-intense}{HTML}{282C36}
    \definecolor{ansi-red}{HTML}{E75C58}
    \definecolor{ansi-red-intense}{HTML}{B22B31}
    \definecolor{ansi-green}{HTML}{00A250}
    \definecolor{ansi-green-intense}{HTML}{007427}
    \definecolor{ansi-yellow}{HTML}{DDB62B}
    \definecolor{ansi-yellow-intense}{HTML}{B27D12}
    \definecolor{ansi-blue}{HTML}{208FFB}
    \definecolor{ansi-blue-intense}{HTML}{0065CA}
    \definecolor{ansi-magenta}{HTML}{D160C4}
    \definecolor{ansi-magenta-intense}{HTML}{A03196}
    \definecolor{ansi-cyan}{HTML}{60C6C8}
    \definecolor{ansi-cyan-intense}{HTML}{258F8F}
    \definecolor{ansi-white}{HTML}{C5C1B4}
    \definecolor{ansi-white-intense}{HTML}{A1A6B2}

    % commands and environments needed by pandoc snippets
    % extracted from the output of `pandoc -s`
    \providecommand{\tightlist}{%
      \setlength{\itemsep}{0pt}\setlength{\parskip}{0pt}}
    \DefineVerbatimEnvironment{Highlighting}{Verbatim}{commandchars=\\\{\}}
    % Add ',fontsize=\small' for more characters per line
    \newenvironment{Shaded}{}{}
    \newcommand{\KeywordTok}[1]{\textcolor[rgb]{0.00,0.44,0.13}{\textbf{{#1}}}}
    \newcommand{\DataTypeTok}[1]{\textcolor[rgb]{0.56,0.13,0.00}{{#1}}}
    \newcommand{\DecValTok}[1]{\textcolor[rgb]{0.25,0.63,0.44}{{#1}}}
    \newcommand{\BaseNTok}[1]{\textcolor[rgb]{0.25,0.63,0.44}{{#1}}}
    \newcommand{\FloatTok}[1]{\textcolor[rgb]{0.25,0.63,0.44}{{#1}}}
    \newcommand{\CharTok}[1]{\textcolor[rgb]{0.25,0.44,0.63}{{#1}}}
    \newcommand{\StringTok}[1]{\textcolor[rgb]{0.25,0.44,0.63}{{#1}}}
    \newcommand{\CommentTok}[1]{\textcolor[rgb]{0.38,0.63,0.69}{\textit{{#1}}}}
    \newcommand{\OtherTok}[1]{\textcolor[rgb]{0.00,0.44,0.13}{{#1}}}
    \newcommand{\AlertTok}[1]{\textcolor[rgb]{1.00,0.00,0.00}{\textbf{{#1}}}}
    \newcommand{\FunctionTok}[1]{\textcolor[rgb]{0.02,0.16,0.49}{{#1}}}
    \newcommand{\RegionMarkerTok}[1]{{#1}}
    \newcommand{\ErrorTok}[1]{\textcolor[rgb]{1.00,0.00,0.00}{\textbf{{#1}}}}
    \newcommand{\NormalTok}[1]{{#1}}
    
    % Additional commands for more recent versions of Pandoc
    \newcommand{\ConstantTok}[1]{\textcolor[rgb]{0.53,0.00,0.00}{{#1}}}
    \newcommand{\SpecialCharTok}[1]{\textcolor[rgb]{0.25,0.44,0.63}{{#1}}}
    \newcommand{\VerbatimStringTok}[1]{\textcolor[rgb]{0.25,0.44,0.63}{{#1}}}
    \newcommand{\SpecialStringTok}[1]{\textcolor[rgb]{0.73,0.40,0.53}{{#1}}}
    \newcommand{\ImportTok}[1]{{#1}}
    \newcommand{\DocumentationTok}[1]{\textcolor[rgb]{0.73,0.13,0.13}{\textit{{#1}}}}
    \newcommand{\AnnotationTok}[1]{\textcolor[rgb]{0.38,0.63,0.69}{\textbf{\textit{{#1}}}}}
    \newcommand{\CommentVarTok}[1]{\textcolor[rgb]{0.38,0.63,0.69}{\textbf{\textit{{#1}}}}}
    \newcommand{\VariableTok}[1]{\textcolor[rgb]{0.10,0.09,0.49}{{#1}}}
    \newcommand{\ControlFlowTok}[1]{\textcolor[rgb]{0.00,0.44,0.13}{\textbf{{#1}}}}
    \newcommand{\OperatorTok}[1]{\textcolor[rgb]{0.40,0.40,0.40}{{#1}}}
    \newcommand{\BuiltInTok}[1]{{#1}}
    \newcommand{\ExtensionTok}[1]{{#1}}
    \newcommand{\PreprocessorTok}[1]{\textcolor[rgb]{0.74,0.48,0.00}{{#1}}}
    \newcommand{\AttributeTok}[1]{\textcolor[rgb]{0.49,0.56,0.16}{{#1}}}
    \newcommand{\InformationTok}[1]{\textcolor[rgb]{0.38,0.63,0.69}{\textbf{\textit{{#1}}}}}
    \newcommand{\WarningTok}[1]{\textcolor[rgb]{0.38,0.63,0.69}{\textbf{\textit{{#1}}}}}
    
    
    % Define a nice break command that doesn't care if a line doesn't already
    % exist.
    \def\br{\hspace*{\fill} \\* }
    % Math Jax compatability definitions
    \def\gt{>}
    \def\lt{<}
    % Document parameters
    \title{L01 - python basics}
    
    
    

    % Pygments definitions
    
\makeatletter
\def\PY@reset{\let\PY@it=\relax \let\PY@bf=\relax%
    \let\PY@ul=\relax \let\PY@tc=\relax%
    \let\PY@bc=\relax \let\PY@ff=\relax}
\def\PY@tok#1{\csname PY@tok@#1\endcsname}
\def\PY@toks#1+{\ifx\relax#1\empty\else%
    \PY@tok{#1}\expandafter\PY@toks\fi}
\def\PY@do#1{\PY@bc{\PY@tc{\PY@ul{%
    \PY@it{\PY@bf{\PY@ff{#1}}}}}}}
\def\PY#1#2{\PY@reset\PY@toks#1+\relax+\PY@do{#2}}

\expandafter\def\csname PY@tok@w\endcsname{\def\PY@tc##1{\textcolor[rgb]{0.73,0.73,0.73}{##1}}}
\expandafter\def\csname PY@tok@c\endcsname{\let\PY@it=\textit\def\PY@tc##1{\textcolor[rgb]{0.25,0.50,0.50}{##1}}}
\expandafter\def\csname PY@tok@cp\endcsname{\def\PY@tc##1{\textcolor[rgb]{0.74,0.48,0.00}{##1}}}
\expandafter\def\csname PY@tok@k\endcsname{\let\PY@bf=\textbf\def\PY@tc##1{\textcolor[rgb]{0.00,0.50,0.00}{##1}}}
\expandafter\def\csname PY@tok@kp\endcsname{\def\PY@tc##1{\textcolor[rgb]{0.00,0.50,0.00}{##1}}}
\expandafter\def\csname PY@tok@kt\endcsname{\def\PY@tc##1{\textcolor[rgb]{0.69,0.00,0.25}{##1}}}
\expandafter\def\csname PY@tok@o\endcsname{\def\PY@tc##1{\textcolor[rgb]{0.40,0.40,0.40}{##1}}}
\expandafter\def\csname PY@tok@ow\endcsname{\let\PY@bf=\textbf\def\PY@tc##1{\textcolor[rgb]{0.67,0.13,1.00}{##1}}}
\expandafter\def\csname PY@tok@nb\endcsname{\def\PY@tc##1{\textcolor[rgb]{0.00,0.50,0.00}{##1}}}
\expandafter\def\csname PY@tok@nf\endcsname{\def\PY@tc##1{\textcolor[rgb]{0.00,0.00,1.00}{##1}}}
\expandafter\def\csname PY@tok@nc\endcsname{\let\PY@bf=\textbf\def\PY@tc##1{\textcolor[rgb]{0.00,0.00,1.00}{##1}}}
\expandafter\def\csname PY@tok@nn\endcsname{\let\PY@bf=\textbf\def\PY@tc##1{\textcolor[rgb]{0.00,0.00,1.00}{##1}}}
\expandafter\def\csname PY@tok@ne\endcsname{\let\PY@bf=\textbf\def\PY@tc##1{\textcolor[rgb]{0.82,0.25,0.23}{##1}}}
\expandafter\def\csname PY@tok@nv\endcsname{\def\PY@tc##1{\textcolor[rgb]{0.10,0.09,0.49}{##1}}}
\expandafter\def\csname PY@tok@no\endcsname{\def\PY@tc##1{\textcolor[rgb]{0.53,0.00,0.00}{##1}}}
\expandafter\def\csname PY@tok@nl\endcsname{\def\PY@tc##1{\textcolor[rgb]{0.63,0.63,0.00}{##1}}}
\expandafter\def\csname PY@tok@ni\endcsname{\let\PY@bf=\textbf\def\PY@tc##1{\textcolor[rgb]{0.60,0.60,0.60}{##1}}}
\expandafter\def\csname PY@tok@na\endcsname{\def\PY@tc##1{\textcolor[rgb]{0.49,0.56,0.16}{##1}}}
\expandafter\def\csname PY@tok@nt\endcsname{\let\PY@bf=\textbf\def\PY@tc##1{\textcolor[rgb]{0.00,0.50,0.00}{##1}}}
\expandafter\def\csname PY@tok@nd\endcsname{\def\PY@tc##1{\textcolor[rgb]{0.67,0.13,1.00}{##1}}}
\expandafter\def\csname PY@tok@s\endcsname{\def\PY@tc##1{\textcolor[rgb]{0.73,0.13,0.13}{##1}}}
\expandafter\def\csname PY@tok@sd\endcsname{\let\PY@it=\textit\def\PY@tc##1{\textcolor[rgb]{0.73,0.13,0.13}{##1}}}
\expandafter\def\csname PY@tok@si\endcsname{\let\PY@bf=\textbf\def\PY@tc##1{\textcolor[rgb]{0.73,0.40,0.53}{##1}}}
\expandafter\def\csname PY@tok@se\endcsname{\let\PY@bf=\textbf\def\PY@tc##1{\textcolor[rgb]{0.73,0.40,0.13}{##1}}}
\expandafter\def\csname PY@tok@sr\endcsname{\def\PY@tc##1{\textcolor[rgb]{0.73,0.40,0.53}{##1}}}
\expandafter\def\csname PY@tok@ss\endcsname{\def\PY@tc##1{\textcolor[rgb]{0.10,0.09,0.49}{##1}}}
\expandafter\def\csname PY@tok@sx\endcsname{\def\PY@tc##1{\textcolor[rgb]{0.00,0.50,0.00}{##1}}}
\expandafter\def\csname PY@tok@m\endcsname{\def\PY@tc##1{\textcolor[rgb]{0.40,0.40,0.40}{##1}}}
\expandafter\def\csname PY@tok@gh\endcsname{\let\PY@bf=\textbf\def\PY@tc##1{\textcolor[rgb]{0.00,0.00,0.50}{##1}}}
\expandafter\def\csname PY@tok@gu\endcsname{\let\PY@bf=\textbf\def\PY@tc##1{\textcolor[rgb]{0.50,0.00,0.50}{##1}}}
\expandafter\def\csname PY@tok@gd\endcsname{\def\PY@tc##1{\textcolor[rgb]{0.63,0.00,0.00}{##1}}}
\expandafter\def\csname PY@tok@gi\endcsname{\def\PY@tc##1{\textcolor[rgb]{0.00,0.63,0.00}{##1}}}
\expandafter\def\csname PY@tok@gr\endcsname{\def\PY@tc##1{\textcolor[rgb]{1.00,0.00,0.00}{##1}}}
\expandafter\def\csname PY@tok@ge\endcsname{\let\PY@it=\textit}
\expandafter\def\csname PY@tok@gs\endcsname{\let\PY@bf=\textbf}
\expandafter\def\csname PY@tok@gp\endcsname{\let\PY@bf=\textbf\def\PY@tc##1{\textcolor[rgb]{0.00,0.00,0.50}{##1}}}
\expandafter\def\csname PY@tok@go\endcsname{\def\PY@tc##1{\textcolor[rgb]{0.53,0.53,0.53}{##1}}}
\expandafter\def\csname PY@tok@gt\endcsname{\def\PY@tc##1{\textcolor[rgb]{0.00,0.27,0.87}{##1}}}
\expandafter\def\csname PY@tok@err\endcsname{\def\PY@bc##1{\setlength{\fboxsep}{0pt}\fcolorbox[rgb]{1.00,0.00,0.00}{1,1,1}{\strut ##1}}}
\expandafter\def\csname PY@tok@kc\endcsname{\let\PY@bf=\textbf\def\PY@tc##1{\textcolor[rgb]{0.00,0.50,0.00}{##1}}}
\expandafter\def\csname PY@tok@kd\endcsname{\let\PY@bf=\textbf\def\PY@tc##1{\textcolor[rgb]{0.00,0.50,0.00}{##1}}}
\expandafter\def\csname PY@tok@kn\endcsname{\let\PY@bf=\textbf\def\PY@tc##1{\textcolor[rgb]{0.00,0.50,0.00}{##1}}}
\expandafter\def\csname PY@tok@kr\endcsname{\let\PY@bf=\textbf\def\PY@tc##1{\textcolor[rgb]{0.00,0.50,0.00}{##1}}}
\expandafter\def\csname PY@tok@bp\endcsname{\def\PY@tc##1{\textcolor[rgb]{0.00,0.50,0.00}{##1}}}
\expandafter\def\csname PY@tok@fm\endcsname{\def\PY@tc##1{\textcolor[rgb]{0.00,0.00,1.00}{##1}}}
\expandafter\def\csname PY@tok@vc\endcsname{\def\PY@tc##1{\textcolor[rgb]{0.10,0.09,0.49}{##1}}}
\expandafter\def\csname PY@tok@vg\endcsname{\def\PY@tc##1{\textcolor[rgb]{0.10,0.09,0.49}{##1}}}
\expandafter\def\csname PY@tok@vi\endcsname{\def\PY@tc##1{\textcolor[rgb]{0.10,0.09,0.49}{##1}}}
\expandafter\def\csname PY@tok@vm\endcsname{\def\PY@tc##1{\textcolor[rgb]{0.10,0.09,0.49}{##1}}}
\expandafter\def\csname PY@tok@sa\endcsname{\def\PY@tc##1{\textcolor[rgb]{0.73,0.13,0.13}{##1}}}
\expandafter\def\csname PY@tok@sb\endcsname{\def\PY@tc##1{\textcolor[rgb]{0.73,0.13,0.13}{##1}}}
\expandafter\def\csname PY@tok@sc\endcsname{\def\PY@tc##1{\textcolor[rgb]{0.73,0.13,0.13}{##1}}}
\expandafter\def\csname PY@tok@dl\endcsname{\def\PY@tc##1{\textcolor[rgb]{0.73,0.13,0.13}{##1}}}
\expandafter\def\csname PY@tok@s2\endcsname{\def\PY@tc##1{\textcolor[rgb]{0.73,0.13,0.13}{##1}}}
\expandafter\def\csname PY@tok@sh\endcsname{\def\PY@tc##1{\textcolor[rgb]{0.73,0.13,0.13}{##1}}}
\expandafter\def\csname PY@tok@s1\endcsname{\def\PY@tc##1{\textcolor[rgb]{0.73,0.13,0.13}{##1}}}
\expandafter\def\csname PY@tok@mb\endcsname{\def\PY@tc##1{\textcolor[rgb]{0.40,0.40,0.40}{##1}}}
\expandafter\def\csname PY@tok@mf\endcsname{\def\PY@tc##1{\textcolor[rgb]{0.40,0.40,0.40}{##1}}}
\expandafter\def\csname PY@tok@mh\endcsname{\def\PY@tc##1{\textcolor[rgb]{0.40,0.40,0.40}{##1}}}
\expandafter\def\csname PY@tok@mi\endcsname{\def\PY@tc##1{\textcolor[rgb]{0.40,0.40,0.40}{##1}}}
\expandafter\def\csname PY@tok@il\endcsname{\def\PY@tc##1{\textcolor[rgb]{0.40,0.40,0.40}{##1}}}
\expandafter\def\csname PY@tok@mo\endcsname{\def\PY@tc##1{\textcolor[rgb]{0.40,0.40,0.40}{##1}}}
\expandafter\def\csname PY@tok@ch\endcsname{\let\PY@it=\textit\def\PY@tc##1{\textcolor[rgb]{0.25,0.50,0.50}{##1}}}
\expandafter\def\csname PY@tok@cm\endcsname{\let\PY@it=\textit\def\PY@tc##1{\textcolor[rgb]{0.25,0.50,0.50}{##1}}}
\expandafter\def\csname PY@tok@cpf\endcsname{\let\PY@it=\textit\def\PY@tc##1{\textcolor[rgb]{0.25,0.50,0.50}{##1}}}
\expandafter\def\csname PY@tok@c1\endcsname{\let\PY@it=\textit\def\PY@tc##1{\textcolor[rgb]{0.25,0.50,0.50}{##1}}}
\expandafter\def\csname PY@tok@cs\endcsname{\let\PY@it=\textit\def\PY@tc##1{\textcolor[rgb]{0.25,0.50,0.50}{##1}}}

\def\PYZbs{\char`\\}
\def\PYZus{\char`\_}
\def\PYZob{\char`\{}
\def\PYZcb{\char`\}}
\def\PYZca{\char`\^}
\def\PYZam{\char`\&}
\def\PYZlt{\char`\<}
\def\PYZgt{\char`\>}
\def\PYZsh{\char`\#}
\def\PYZpc{\char`\%}
\def\PYZdl{\char`\$}
\def\PYZhy{\char`\-}
\def\PYZsq{\char`\'}
\def\PYZdq{\char`\"}
\def\PYZti{\char`\~}
% for compatibility with earlier versions
\def\PYZat{@}
\def\PYZlb{[}
\def\PYZrb{]}
\makeatother


    % Exact colors from NB
    \definecolor{incolor}{rgb}{0.0, 0.0, 0.5}
    \definecolor{outcolor}{rgb}{0.545, 0.0, 0.0}



    
    % Prevent overflowing lines due to hard-to-break entities
    \sloppy 
    % Setup hyperref package
    \hypersetup{
      breaklinks=true,  % so long urls are correctly broken across lines
      colorlinks=true,
      urlcolor=urlcolor,
      linkcolor=linkcolor,
      citecolor=citecolor,
      }
    % Slightly bigger margins than the latex defaults
    
    \geometry{verbose,tmargin=1in,bmargin=1in,lmargin=1in,rmargin=1in}
    
    

    \begin{document}
    
    
    \maketitle
    
    

    
    \textbf{Author: Nico Grisouard, nicolas.grisouard@physics.utoronto.ca}

\emph{Supporting textbook chapters for week 1: 2, 3 and 4.3}

This is an example of ``lecture notes''. As you will quickly find out,
this course is by nature a lab course. Therefore, my ``lecture notes''
will not often follow the linear progression of regular lecture notes.
This is particularly true for this first lecture, in which I merely want
to give you pointers to do the first lab.

    \hypertarget{machine-error-round-off-error.}{%
\section{Machine error: round-off
error.}\label{machine-error-round-off-error.}}

Under what circumstances is the following possible?
\[(x+y)+z \neq x + (y + z)\]

Let's try it in Python:

    \begin{Verbatim}[commandchars=\\\{\}]
{\color{incolor}In [{\color{incolor}1}]:} \PY{n}{x} \PY{o}{=} \PY{l+m+mf}{1e20}
        \PY{n}{y} \PY{o}{=} \PY{o}{\PYZhy{}}\PY{l+m+mf}{1e20}
        \PY{n}{z} \PY{o}{=} \PY{l+m+mf}{1.}
\end{Verbatim}


    \begin{Verbatim}[commandchars=\\\{\}]
{\color{incolor}In [{\color{incolor}2}]:} \PY{p}{(}\PY{n}{x}\PY{o}{+}\PY{n}{y}\PY{p}{)}\PY{o}{+}\PY{n}{z}
\end{Verbatim}


\begin{Verbatim}[commandchars=\\\{\}]
{\color{outcolor}Out[{\color{outcolor}2}]:} 1.0
\end{Verbatim}
            
    \begin{Verbatim}[commandchars=\\\{\}]
{\color{incolor}In [{\color{incolor}3}]:} \PY{n}{x}\PY{o}{+}\PY{p}{(}\PY{n}{y}\PY{o}{+}\PY{n}{z}\PY{p}{)}
\end{Verbatim}


\begin{Verbatim}[commandchars=\\\{\}]
{\color{outcolor}Out[{\color{outcolor}3}]:} 0.0
\end{Verbatim}
            
    What happened?

    Round-off error!

    \hypertarget{algorithmic-error-instability}{%
\section{Algorithmic error:
instability}\label{algorithmic-error-instability}}

Consider this system representing phasor rotation in the complex plane:
\[\dot Z = i\omega Z, \quad\text{given}\quad Z_0 = Z(t=0).\] Solution is
\(Z(t) = Z_0 \exp(i\omega t).\)

How can we solve it numerically?

    Taylor expansion:
\[\dot Z(t) = \frac{Z(t+\Delta t)-Z(t)}{\Delta t} + H.O.T. = i\omega Z(t).\]

Suggests algorithm: * Start with \(Z(t=0) = Z_{old}\), *
\(Z_{new} = (1+i\omega \Delta t)Z_{old}\), * repeat.

    \textbf{Let's code it up and see}

    \begin{Verbatim}[commandchars=\\\{\}]
{\color{incolor}In [{\color{incolor} }]:} \PY{c+c1}{\PYZsh{} Unstable solution to dz/dt}
        \PY{k+kn}{from} \PY{n+nn}{math} \PY{k}{import} \PY{n}{pi}
        
        \PY{n}{Z} \PY{o}{=} \PY{n+nb}{complex}\PY{p}{(}\PY{l+m+mf}{1.0}\PY{p}{,} \PY{l+m+mf}{0.0}\PY{p}{)}  \PY{c+c1}{\PYZsh{} z0 initial condition}
        \PY{n}{omega} \PY{o}{=} \PY{l+m+mf}{1.0}  \PY{c+c1}{\PYZsh{} angular frequency 2*pi/period}
        \PY{n}{dt} \PY{o}{=} \PY{l+m+mi}{2}\PY{o}{*}\PY{n}{pi}\PY{o}{/}\PY{p}{(}\PY{l+m+mi}{200}\PY{o}{*}\PY{n}{omega}\PY{p}{)}  \PY{c+c1}{\PYZsh{} time step is 200 per period}
        \PY{c+c1}{\PYZsh{} print initial information}
        \PY{n+nb}{print}\PY{p}{(}\PY{l+s+s1}{\PYZsq{}}\PY{l+s+s1}{t = }\PY{l+s+si}{\PYZob{}0\PYZcb{}}\PY{l+s+s1}{, Z = }\PY{l+s+si}{\PYZob{}1\PYZcb{}}\PY{l+s+s1}{, |Z| = }\PY{l+s+si}{\PYZob{}2\PYZcb{}}\PY{l+s+s1}{\PYZsq{}}\PY{o}{.}\PY{n}{format}\PY{p}{(}\PY{l+m+mf}{0.0}\PY{p}{,} \PY{n}{Z}\PY{p}{,} \PY{n+nb}{abs}\PY{p}{(}\PY{n}{Z}\PY{p}{)}\PY{p}{)}\PY{p}{)}
\end{Verbatim}


    \begin{Verbatim}[commandchars=\\\{\}]
{\color{incolor}In [{\color{incolor} }]:} \PY{k}{for} \PY{n}{k} \PY{o+ow}{in} \PY{n+nb}{range}\PY{p}{(}\PY{l+m+mi}{201}\PY{p}{)}\PY{p}{:}  \PY{c+c1}{\PYZsh{} update z}
            \PY{n}{Z} \PY{o}{*}\PY{o}{=} \PY{n+nb}{complex}\PY{p}{(}\PY{l+m+mf}{1.0}\PY{p}{,} \PY{n}{dt}\PY{p}{)}  \PY{c+c1}{\PYZsh{} complex(1.0, dt) is 1+i*dt}
        
        \PY{c+c1}{\PYZsh{} print final information}
        \PY{n+nb}{print}\PY{p}{(}\PY{l+s+s1}{\PYZsq{}}\PY{l+s+s1}{t = }\PY{l+s+si}{\PYZob{}0\PYZcb{}}\PY{l+s+s1}{\PYZsq{}}\PY{o}{.}\PY{n}{format}\PY{p}{(}\PY{n}{k}\PY{o}{*}\PY{n}{dt}\PY{p}{)}\PY{p}{)}
        \PY{n+nb}{print}\PY{p}{(}\PY{l+s+s1}{\PYZsq{}}\PY{l+s+s1}{Z = }\PY{l+s+si}{\PYZob{}0\PYZcb{}}\PY{l+s+s1}{\PYZsq{}}\PY{o}{.}\PY{n}{format}\PY{p}{(}\PY{n}{Z}\PY{p}{)}\PY{p}{)}
        \PY{n+nb}{print}\PY{p}{(}\PY{l+s+s1}{\PYZsq{}}\PY{l+s+s1}{|Z| = }\PY{l+s+si}{\PYZob{}0\PYZcb{}}\PY{l+s+s1}{\PYZsq{}}\PY{o}{.}\PY{n}{format}\PY{p}{(}\PY{n+nb}{abs}\PY{p}{(}\PY{n}{Z}\PY{p}{)}\PY{p}{)}\PY{p}{)}
\end{Verbatim}


    What happened? What is the problem? Why did it happen?

    Next time!

    \hypertarget{typical-approach-for-solving-a-problem}{%
\section{Typical approach for solving a
problem}\label{typical-approach-for-solving-a-problem}}

    \begin{enumerate}
\def\labelenumi{\arabic{enumi}.}
\tightlist
\item
  Start with math model, often but not always continuous.
\item
  \textbf{Discretize:} set up discrete arrays of independent variables
  (e.g., \(x\), \(t\)), dependent variables (e.g. \(v(t)\), \(a(t)\)),
  and define operators on these variables (\(dv/dt\), \(ma\)\ldots{}).
\item
  \textbf{Initialize} parameters and variables appropriately.
\item
  \textbf{Evalutize:} run algorithm to operate on these variables.
\item
  \textbf{Analyze:} some extra processing of the raw results,
  figures\ldots{}
\end{enumerate}

See example:

    \begin{Verbatim}[commandchars=\\\{\}]
{\color{incolor}In [{\color{incolor} }]:} \PY{c+c1}{\PYZsh{} Example to illustrate general procedure for mathematical modeling.}
        \PY{c+c1}{\PYZsh{} Given position x(t) of a particle undergoing SHO,}
        \PY{c+c1}{\PYZsh{} calculate velocity and acceleration using simple finite difference}
        \PY{k+kn}{import} \PY{n+nn}{numpy} \PY{k}{as} \PY{n+nn}{np}  \PY{c+c1}{\PYZsh{} import numpy}
        \PY{k+kn}{import} \PY{n+nn}{matplotlib}\PY{n+nn}{.}\PY{n+nn}{pyplot} \PY{k}{as} \PY{n+nn}{plt}  \PY{c+c1}{\PYZsh{} import figure functions}
\end{Verbatim}


    \begin{Verbatim}[commandchars=\\\{\}]
{\color{incolor}In [{\color{incolor} }]:} \PY{c+c1}{\PYZsh{} 2. Discretize}
        
        \PY{c+c1}{\PYZsh{}Define time grid and dependent variables}
        \PY{n}{t} \PY{o}{=} \PY{n}{np}\PY{o}{.}\PY{n}{linspace}\PY{p}{(}\PY{l+m+mi}{0}\PY{p}{,} \PY{l+m+mi}{10}\PY{p}{,} \PY{l+m+mi}{101}\PY{p}{)}  \PY{c+c1}{\PYZsh{} time array}
        \PY{n}{N} \PY{o}{=} \PY{n+nb}{len}\PY{p}{(}\PY{n}{t}\PY{p}{)}  \PY{c+c1}{\PYZsh{} number of time steps}
        \PY{n}{x} \PY{o}{=} \PY{n}{np}\PY{o}{.}\PY{n}{zeros}\PY{p}{(}\PY{n}{N}\PY{p}{)}  \PY{c+c1}{\PYZsh{} array of positions}
        \PY{n}{v} \PY{o}{=} \PY{n}{np}\PY{o}{.}\PY{n}{zeros}\PY{p}{(}\PY{n}{N}\PY{o}{\PYZhy{}}\PY{l+m+mi}{1}\PY{p}{)}  \PY{c+c1}{\PYZsh{} array of velocities}
        \PY{n}{a} \PY{o}{=} \PY{n}{np}\PY{o}{.}\PY{n}{zeros}\PY{p}{(}\PY{n}{N}\PY{o}{\PYZhy{}}\PY{l+m+mi}{2}\PY{p}{)}  \PY{c+c1}{\PYZsh{} array of accelerations}
\end{Verbatim}


    \begin{Verbatim}[commandchars=\\\{\}]
{\color{incolor}In [{\color{incolor} }]:} \PY{c+c1}{\PYZsh{} 3. Initialize: define signal on discretized grid}
        \PY{n}{x} \PY{o}{=} \PY{l+m+mf}{3.0}\PY{o}{*}\PY{n}{np}\PY{o}{.}\PY{n}{sin}\PY{p}{(}\PY{n}{t}\PY{p}{)}
\end{Verbatim}


    \begin{Verbatim}[commandchars=\\\{\}]
{\color{incolor}In [{\color{incolor} }]:} \PY{c+c1}{\PYZsh{} 2. discretize and 4. evalutize (here, apply algorithm)}
        \PY{c+c1}{\PYZsh{} Define velocity using finite differences: v = Delta x/Delta t}
        \PY{k}{for} \PY{n}{k} \PY{o+ow}{in} \PY{n+nb}{range}\PY{p}{(}\PY{n+nb}{len}\PY{p}{(}\PY{n}{x}\PY{p}{)}\PY{o}{\PYZhy{}}\PY{l+m+mi}{1}\PY{p}{)}\PY{p}{:}
            \PY{n}{v}\PY{p}{[}\PY{n}{k}\PY{p}{]} \PY{o}{=} \PY{p}{(}\PY{n}{x}\PY{p}{[}\PY{n}{k}\PY{o}{+}\PY{l+m+mi}{1}\PY{p}{]}\PY{o}{\PYZhy{}}\PY{n}{x}\PY{p}{[}\PY{n}{k}\PY{p}{]}\PY{p}{)}\PY{o}{/}\PY{p}{(}\PY{n}{t}\PY{p}{[}\PY{n}{k}\PY{o}{+}\PY{l+m+mi}{1}\PY{p}{]}\PY{o}{\PYZhy{}}\PY{n}{t}\PY{p}{[}\PY{n}{k}\PY{p}{]}\PY{p}{)}
        
        \PY{c+c1}{\PYZsh{} Define acceleration using finite differences: a = Delta v/Delta t}
        \PY{k}{for} \PY{n}{k} \PY{o+ow}{in} \PY{n+nb}{range}\PY{p}{(}\PY{n+nb}{len}\PY{p}{(}\PY{n}{x}\PY{p}{)}\PY{o}{\PYZhy{}}\PY{l+m+mi}{2}\PY{p}{)}\PY{p}{:}
            \PY{n}{a}\PY{p}{[}\PY{n}{k}\PY{p}{]} \PY{o}{=} \PY{p}{(}\PY{n}{v}\PY{p}{[}\PY{n}{k}\PY{o}{+}\PY{l+m+mi}{1}\PY{p}{]}\PY{o}{\PYZhy{}}\PY{n}{v}\PY{p}{[}\PY{n}{k}\PY{p}{]}\PY{p}{)}\PY{o}{/}\PY{p}{(}\PY{n}{t}\PY{p}{[}\PY{n}{k}\PY{o}{+}\PY{l+m+mi}{1}\PY{p}{]}\PY{o}{\PYZhy{}}\PY{n}{t}\PY{p}{[}\PY{n}{k}\PY{p}{]}\PY{p}{)}
\end{Verbatim}


    \begin{Verbatim}[commandchars=\\\{\}]
{\color{incolor}In [{\color{incolor} }]:} \PY{c+c1}{\PYZsh{} 5. Analyze}
        \PY{c+c1}{\PYZsh{} print results}
        \PY{n+nb}{print}\PY{p}{(}\PY{l+s+s2}{\PYZdq{}}\PY{l+s+s2}{t is }\PY{l+s+s2}{\PYZdq{}}\PY{p}{,} \PY{n}{t}\PY{p}{)}
\end{Verbatim}


    \begin{Verbatim}[commandchars=\\\{\}]
{\color{incolor}In [{\color{incolor} }]:} \PY{n+nb}{print}\PY{p}{(}\PY{l+s+s2}{\PYZdq{}}\PY{l+s+s2}{x is }\PY{l+s+s2}{\PYZdq{}}\PY{p}{,} \PY{n}{x}\PY{p}{)}
\end{Verbatim}


    \begin{Verbatim}[commandchars=\\\{\}]
{\color{incolor}In [{\color{incolor} }]:} \PY{n+nb}{print}\PY{p}{(}\PY{l+s+s2}{\PYZdq{}}\PY{l+s+s2}{a is }\PY{l+s+s2}{\PYZdq{}}\PY{p}{,} \PY{n}{a}\PY{p}{)}
\end{Verbatim}


    \begin{Verbatim}[commandchars=\\\{\}]
{\color{incolor}In [{\color{incolor} }]:} \PY{c+c1}{\PYZsh{} plot results}
        \PY{n}{plt}\PY{o}{.}\PY{n}{figure}\PY{p}{(}\PY{p}{)}
        
        \PY{n}{plt}\PY{o}{.}\PY{n}{subplot}\PY{p}{(}\PY{l+m+mi}{3}\PY{p}{,} \PY{l+m+mi}{1}\PY{p}{,} \PY{l+m+mi}{1}\PY{p}{)}
        \PY{n}{plt}\PY{o}{.}\PY{n}{plot}\PY{p}{(}\PY{n}{t}\PY{p}{,} \PY{n}{x}\PY{p}{)}
        \PY{n}{plt}\PY{o}{.}\PY{n}{xlabel}\PY{p}{(}\PY{l+s+s1}{\PYZsq{}}\PY{l+s+s1}{t}\PY{l+s+s1}{\PYZsq{}}\PY{p}{)}
        \PY{n}{plt}\PY{o}{.}\PY{n}{ylabel}\PY{p}{(}\PY{l+s+s1}{\PYZsq{}}\PY{l+s+s1}{x}\PY{l+s+s1}{\PYZsq{}}\PY{p}{)}
        
        \PY{n}{plt}\PY{o}{.}\PY{n}{subplot}\PY{p}{(}\PY{l+m+mi}{3}\PY{p}{,} \PY{l+m+mi}{1}\PY{p}{,} \PY{l+m+mi}{2}\PY{p}{)}
        \PY{n}{plt}\PY{o}{.}\PY{n}{plot}\PY{p}{(}\PY{n}{t}\PY{p}{[}\PY{p}{:}\PY{n}{N}\PY{o}{\PYZhy{}}\PY{l+m+mi}{1}\PY{p}{]}\PY{p}{,} \PY{n}{v}\PY{p}{)}
        \PY{n}{plt}\PY{o}{.}\PY{n}{xlabel}\PY{p}{(}\PY{l+s+s1}{\PYZsq{}}\PY{l+s+s1}{t}\PY{l+s+s1}{\PYZsq{}}\PY{p}{)}
        \PY{n}{plt}\PY{o}{.}\PY{n}{ylabel}\PY{p}{(}\PY{l+s+s1}{\PYZsq{}}\PY{l+s+s1}{v}\PY{l+s+s1}{\PYZsq{}}\PY{p}{)}
        
        \PY{n}{plt}\PY{o}{.}\PY{n}{subplot}\PY{p}{(}\PY{l+m+mi}{3}\PY{p}{,} \PY{l+m+mi}{1}\PY{p}{,} \PY{l+m+mi}{3}\PY{p}{)}
        \PY{n}{plt}\PY{o}{.}\PY{n}{plot}\PY{p}{(}\PY{n}{t}\PY{p}{[}\PY{p}{:}\PY{n}{N}\PY{o}{\PYZhy{}}\PY{l+m+mi}{2}\PY{p}{]}\PY{p}{,} \PY{n}{a}\PY{p}{)}
        \PY{n}{plt}\PY{o}{.}\PY{n}{xlabel}\PY{p}{(}\PY{l+s+s1}{\PYZsq{}}\PY{l+s+s1}{t}\PY{l+s+s1}{\PYZsq{}}\PY{p}{)}
        \PY{n}{plt}\PY{o}{.}\PY{n}{ylabel}\PY{p}{(}\PY{l+s+s1}{\PYZsq{}}\PY{l+s+s1}{a}\PY{l+s+s1}{\PYZsq{}}\PY{p}{)}
        
        \PY{c+c1}{\PYZsh{} plt.savefig(\PYZsq{}T01.pdf\PYZsq{})  \PYZsh{} saves a pdf figure on disk}
\end{Verbatim}


    \hypertarget{pseudo-code}{%
\section{Pseudo-code}\label{pseudo-code}}

\hypertarget{general-principles}{%
\subsection{General principles}\label{general-principles}}

\begin{itemize}
\tightlist
\item
  pseudocode is the planned version of your code, written in plain
  English (\(\neq\) programming language)
\item
  You should write one before starting any code.
\item
  It should describe your algorithm.
\item
  It helps ensure that your planned logic for the algorithm is sound.
\item
  In the previous example: the enunmeration (points 1-5) was the
  skeleton of one.
\end{itemize}

    \begin{itemize}
\tightlist
\item
  Real text or comments for your real code in the end.
\item
  \textbf{Keep a copy of it intact} so you can refer to it when you are
  coding.
\item
  Coding = your pseudocode \(\to\) specific programming language. Be
  able to take your pseudocode and convert it into any typical
  programming language.
\end{itemize}

    \begin{itemize}
\tightlist
\item
  Pseudocode: somewhat personal. You do you.
\item
  Concise, logical, step-by-step.
\item
  Start with \textbf{brief} overview of what this piece of code will do.
\end{itemize}

Examples for sequential stuff: * Input: \texttt{READ}, \texttt{OBTAIN},
\texttt{GET} * Initialize: \texttt{SET}, \texttt{DEFINE} * Compute:
\texttt{COMPUTE}, \texttt{CALCULATE}, \texttt{DETERMINE} * Add one:
\texttt{INCREMENT}, \texttt{BUMP} * Output: \texttt{PRINT},
\texttt{DISPLAY}, \texttt{PLOT}, \texttt{WRITE}

    Examples for conditions and loops: * \texttt{WHILE},
\texttt{IF-THEN-ELSE}, \texttt{REPEAT-UNTIL}, \texttt{CASE},
\texttt{FOR}

Should also include calling functions: * \texttt{CALL}

    \hypertarget{pseudo-code-example-1}{%
\subsection{Pseudo-code, example 1}\label{pseudo-code-example-1}}

Convert polar to Cartesian coordinates from keyboard input:

\(r\), \(\theta\) (\(^\circ\)) \(\to\) \(x\), \(y\).

Let's try:

    \begin{enumerate}
\def\labelenumi{\arabic{enumi}.}
\tightlist
\item
  From keyboard, read radius \(r\) and save.
\item
  From keyboard, read angle \(\theta\) in degrees and save.
\item
  Do the conversion from degrees to angles
  (\(\theta_r = \pi \theta/180\)).
\item
  Compute \((x, y) = r(\cos\theta_r, \sin\theta_r)\).
\item
  Print result to screen.
\end{enumerate}

    Alternative: write the pseudo-code code comments.

    \begin{Verbatim}[commandchars=\\\{\}]
{\color{incolor}In [{\color{incolor} }]:} \PY{c+c1}{\PYZsh{} Re\PYZhy{}import numpy, in case I am executing cells in random order}
        \PY{k+kn}{import} \PY{n+nn}{numpy} \PY{k}{as} \PY{n+nn}{np}
\end{Verbatim}


    \begin{Verbatim}[commandchars=\\\{\}]
{\color{incolor}In [{\color{incolor} }]:} \PY{c+c1}{\PYZsh{} 1. From keyboard, read the radius and save.}
        \PY{n}{r} \PY{o}{=} \PY{n+nb}{float}\PY{p}{(}\PY{n+nb}{input}\PY{p}{(}\PY{l+s+s2}{\PYZdq{}}\PY{l+s+s2}{Enter r:}\PY{l+s+s2}{\PYZdq{}}\PY{p}{)}\PY{p}{)}
\end{Verbatim}


    \begin{Verbatim}[commandchars=\\\{\}]
{\color{incolor}In [{\color{incolor} }]:} \PY{c+c1}{\PYZsh{} 2. From keyboard, read de angle \PYZdl{}\PYZbs{}theta\PYZdl{} in degrees and save.}
        \PY{n}{theta\PYZus{}deg} \PY{o}{=} \PY{n+nb}{float}\PY{p}{(}\PY{n+nb}{input}\PY{p}{(}\PY{l+s+s2}{\PYZdq{}}\PY{l+s+s2}{Enter theta in degrees:}\PY{l+s+s2}{\PYZdq{}}\PY{p}{)}\PY{p}{)}
\end{Verbatim}


    \begin{Verbatim}[commandchars=\\\{\}]
{\color{incolor}In [{\color{incolor} }]:} \PY{c+c1}{\PYZsh{} 3. Do the conversion from degrees to angles (\PYZdl{}\PYZbs{}theta\PYZus{}r = \PYZbs{}pi \PYZbs{}theta/180\PYZdl{}).}
        \PY{n}{theta\PYZus{}r} \PY{o}{=} \PY{n}{np}\PY{o}{.}\PY{n}{pi}\PY{o}{*}\PY{n}{theta\PYZus{}deg}\PY{o}{/}\PY{l+m+mf}{180.}
\end{Verbatim}


    \begin{Verbatim}[commandchars=\\\{\}]
{\color{incolor}In [{\color{incolor} }]:} \PY{c+c1}{\PYZsh{} 4. Compute \PYZdl{}(x, y) = r(\PYZbs{}cos\PYZbs{}theta\PYZus{}r, \PYZbs{}sin\PYZbs{}theta\PYZus{}r)\PYZdl{}.}
        \PY{n}{x} \PY{o}{=} \PY{n}{r} \PY{o}{*} \PY{n}{np}\PY{o}{.}\PY{n}{cos}\PY{p}{(}\PY{n}{theta\PYZus{}r}\PY{p}{)}
        \PY{n}{y} \PY{o}{=} \PY{n}{r} \PY{o}{*} \PY{n}{np}\PY{o}{.}\PY{n}{sin}\PY{p}{(}\PY{n}{theta\PYZus{}r}\PY{p}{)}
\end{Verbatim}


    \begin{Verbatim}[commandchars=\\\{\}]
{\color{incolor}In [{\color{incolor} }]:} \PY{c+c1}{\PYZsh{} 5. Print result to screen.}
        \PY{n+nb}{print}\PY{p}{(}\PY{l+s+s2}{\PYZdq{}}\PY{l+s+s2}{x = }\PY{l+s+s2}{\PYZdq{}}\PY{p}{,} \PY{n}{x}\PY{p}{,} \PY{l+s+s2}{\PYZdq{}}\PY{l+s+s2}{y = }\PY{l+s+s2}{\PYZdq{}}\PY{p}{,} \PY{n}{y}\PY{p}{)}
\end{Verbatim}


    \hypertarget{pseudo-code-example-2}{%
\subsection{Pseudo-code, example 2}\label{pseudo-code-example-2}}

\begin{itemize}
\tightlist
\item
  Number of radioactive atoms of uranium \(N\) as a function of time
  given initial \(N_0\) and \[\frac{dN}{dt} = -\frac{N}{\tau}.\]
  (\$\tau = \$ decay constant)
\item
  Use Euler method for integration.
\end{itemize}

    \begin{Verbatim}[commandchars=\\\{\}]
{\color{incolor}In [{\color{incolor} }]:} \PY{c+c1}{\PYZsh{} Pseudocode:}
        
        \PY{c+c1}{\PYZsh{} 1. Define decay constant tau, initial N0, and end time 5*tau}
        \PY{c+c1}{\PYZsh{} 2. Initialize time and time array with 101 values}
        \PY{c+c1}{\PYZsh{} 3. Compute dt the time step because we\PYZsq{}ll need it}
        \PY{c+c1}{\PYZsh{} 4. Initialize a number array with same number of particles}
        \PY{c+c1}{\PYZsh{} 5. Increment the N array with Euler: N[i+1] = N[i] \PYZhy{} dt*N[i]/tau}
        \PY{c+c1}{\PYZsh{} 6. Plot N vs. t}
\end{Verbatim}


    \begin{Verbatim}[commandchars=\\\{\}]
{\color{incolor}In [{\color{incolor} }]:} \PY{c+c1}{\PYZsh{} Code that computes the number of radioactive atoms of uranium N}
        \PY{c+c1}{\PYZsh{} as fct of time, given tau and N0 and using Euler method.}
        \PY{c+c1}{\PYZsh{} Author: Nicolas Grisouard}
        
        \PY{c+c1}{\PYZsh{} Re\PYZhy{}import, in case I am executing cells in random order}
        \PY{k+kn}{import} \PY{n+nn}{numpy} \PY{k}{as} \PY{n+nn}{np}
        \PY{k+kn}{import} \PY{n+nn}{matplotlib}\PY{n+nn}{.}\PY{n+nn}{pyplot} \PY{k}{as} \PY{n+nn}{plt}
\end{Verbatim}


    \begin{Verbatim}[commandchars=\\\{\}]
{\color{incolor}In [{\color{incolor} }]:} \PY{c+c1}{\PYZsh{} I skipped this cell in class: it is a conversion of the }
        \PY{c+c1}{\PYZsh{} half\PYZhy{}life of U238 (which is easily found on Wikipedia)}
        \PY{c+c1}{\PYZsh{} and the decay rate (which for some reason is not)}
        \PY{n+nb}{print}\PY{p}{(}\PY{l+s+s2}{\PYZdq{}}\PY{l+s+si}{\PYZob{}0:1.3e\PYZcb{}}\PY{l+s+s2}{\PYZdq{}}\PY{o}{.}\PY{n}{format}\PY{p}{(}\PY{l+m+mf}{4.468e9}\PY{o}{/}\PY{n}{np}\PY{o}{.}\PY{n}{log}\PY{p}{(}\PY{l+m+mf}{2.}\PY{p}{)}\PY{p}{)}\PY{p}{)}
\end{Verbatim}


    \begin{Verbatim}[commandchars=\\\{\}]
{\color{incolor}In [{\color{incolor} }]:} \PY{c+c1}{\PYZsh{} 1. Define decay constant tau, initial N0, and end time 5*tau}
        \PY{n}{tau} \PY{o}{=} \PY{l+m+mf}{6.446e9}  \PY{c+c1}{\PYZsh{} [s] tau for U238}
        \PY{n}{N0} \PY{o}{=} \PY{l+m+mf}{1e9}  \PY{c+c1}{\PYZsh{} [1] why not}
        \PY{n}{t\PYZus{}end} \PY{o}{=} \PY{l+m+mi}{5}\PY{o}{*}\PY{n}{tau}  \PY{c+c1}{\PYZsh{} [s] the end of the simulation time}
\end{Verbatim}


    \begin{Verbatim}[commandchars=\\\{\}]
{\color{incolor}In [{\color{incolor} }]:} \PY{c+c1}{\PYZsh{} 2. Initialize time array with 101 values}
        \PY{n}{t\PYZus{}array} \PY{o}{=} \PY{n}{np}\PY{o}{.}\PY{n}{linspace}\PY{p}{(}\PY{l+m+mi}{0}\PY{p}{,} \PY{n}{t\PYZus{}end}\PY{p}{,} \PY{l+m+mi}{101}\PY{p}{)}
        \PY{c+c1}{\PYZsh{} 3. Compute dt the time step because we\PYZsq{}ll need it}
        \PY{n}{dt} \PY{o}{=} \PY{n}{t\PYZus{}array}\PY{p}{[}\PY{l+m+mi}{1}\PY{p}{]} \PY{o}{\PYZhy{}} \PY{n}{t\PYZus{}array}\PY{p}{[}\PY{l+m+mi}{0}\PY{p}{]}
        \PY{c+c1}{\PYZsh{} 4. Initialize a number array with same number of elements}
        \PY{n}{N\PYZus{}array} \PY{o}{=} \PY{l+m+mi}{0}\PY{o}{*}\PY{n}{t}
        \PY{n}{N\PYZus{}array}\PY{p}{[}\PY{l+m+mi}{0}\PY{p}{]} \PY{o}{=} \PY{n}{N0}
\end{Verbatim}


    \begin{Verbatim}[commandchars=\\\{\}]
{\color{incolor}In [{\color{incolor} }]:} \PY{c+c1}{\PYZsh{} 5. Increment the N array with Euler: N[i+1] = N[i] \PYZhy{} dt*N[i]/tau}
        \PY{k}{for} \PY{n}{i} \PY{o+ow}{in} \PY{n+nb}{range}\PY{p}{(}\PY{n+nb}{len}\PY{p}{(}\PY{n}{t\PYZus{}array}\PY{p}{)}\PY{o}{\PYZhy{}}\PY{l+m+mi}{1}\PY{p}{)}\PY{p}{:}  \PY{c+c1}{\PYZsh{} I end at the second\PYZhy{}to\PYZhy{}last value}
            \PY{n}{N\PYZus{}array}\PY{p}{[}\PY{n}{i}\PY{o}{+}\PY{l+m+mi}{1}\PY{p}{]} \PY{o}{=} \PY{p}{(}\PY{l+m+mf}{1.} \PY{o}{\PYZhy{}} \PY{n}{dt}\PY{o}{/}\PY{n}{tau}\PY{p}{)}\PY{o}{*}\PY{n}{N\PYZus{}array}\PY{p}{[}\PY{n}{i}\PY{p}{]}
\end{Verbatim}


    \begin{Verbatim}[commandchars=\\\{\}]
{\color{incolor}In [{\color{incolor} }]:} \PY{c+c1}{\PYZsh{} 6. Plot N vs. t}
        \PY{n}{plt}\PY{o}{.}\PY{n}{figure}\PY{p}{(}\PY{p}{)}
        \PY{n}{plt}\PY{o}{.}\PY{n}{plot}\PY{p}{(}\PY{n}{t\PYZus{}array}\PY{p}{,} \PY{n}{N\PYZus{}array}\PY{p}{)}
        \PY{n}{plt}\PY{o}{.}\PY{n}{xlabel}\PY{p}{(}\PY{l+s+s1}{\PYZsq{}}\PY{l+s+s1}{t (a)}\PY{l+s+s1}{\PYZsq{}}\PY{p}{)}
        \PY{n}{plt}\PY{o}{.}\PY{n}{ylabel}\PY{p}{(}\PY{l+s+s1}{\PYZsq{}}\PY{l+s+s1}{N}\PY{l+s+s1}{\PYZsq{}}\PY{p}{)}
        \PY{n}{plt}\PY{o}{.}\PY{n}{grid}\PY{p}{(}\PY{p}{)}
\end{Verbatim}



    % Add a bibliography block to the postdoc
    
    
    
    \end{document}
